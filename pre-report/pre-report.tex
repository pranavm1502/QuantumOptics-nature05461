\documentclass[12pt,notitlepage,nofootinbib]{revtex4-1}
%\usepackage[top=0.75in,bottom=0.75in,left=1in,right=1in]{geometry} 
\usepackage{amsmath,amssymb,esint} 
\allowdisplaybreaks[0]
\usepackage[CJKbookmarks]{hyperref}
\usepackage{bm} 
\usepackage{siunitx} 
\usepackage{graphicx} 
\usepackage{color}
\usepackage{braket}
\bibliographystyle{unsrt}

\renewcommand*{\vec}[1]{\bm{#1}} 
\newcommand{\dif}{\,\mathrm d}
\newcommand{\rf}{\text{rf}}
\newcommand\mi{\mathrm{i}}
\newcommand\e{\mathrm{e}} 

\begin{document}
\title{Pre-report for ELE456: \\
``Resolving photon number states in a superconducting circuit''}
\author{Ming Lyu, Elena de la Hoz Lopez-Collado}
\maketitle
This is the pre-report for the course ELE456 about the project on understanding 
the paper\cite{schuster2007resolving}. 

\section{Understanding the paper}
\subsection{The system of the experiment}
The most important result in this paper is the evidence of discreteness of the 
electromagnetic field energy (photon) on a system of coupled superconducting 
qubit and wave-guide cavity. The Hamiltonian in the paper is: 
\begin{equation}\label{eq:diag_jc}
	H_0 = \hbar\omega_r(a^\dag a + 1/2) + \hbar \omega_a\sigma_z/2 + 
	\hbar\chi(a^\dag a + 1/2)\sigma_z
\end{equation}
Which is actually the diagonalized form of Jaynes-Cummings (JC) model, 
with $2\chi = 2g_0^2/\Delta$:
\begin{equation}\label{eq:bare_jc}
 	H_{\text{JC}} = \hbar\omega_r(a^\dag a +1/2) + \hbar\omega_a\sigma_z/2 + 
\hbar g_0(a^\dag\sigma^- + a\sigma^+)
\end{equation}
in the large detuning limit ($\Delta = |\omega_r - \omega_a| \gg g_0$), with 
transformation matrix:
\begin{equation}\label{eq:diag_matrix}
	U = \exp \left[\frac {g_0}\Delta(a\sigma^+ - a^\dag\sigma^-)\right]
\end{equation}
(Eq.~(12) in \cite{blais2004cavity}, $H_0 = UH_{\text{JC}}U^\dag$). 

The Hamiltonian in Eq.~(\ref{eq:diag_jc}) commutes with ``photon number'' 
operator $a^\dag a$ and $\sigma_z$ \footnote{Strictly speaking they are not 
actually photon number and Pauli matrix for the spin, but $U a^\dag a U^\dag$ 
and $U \sigma_z U^\dag$ in JC model, though the difference is small in the 
large detuning limit.}. This means that energy eigenstates are also eigenstates 
of ``photon number'' and spin state. For given ``photon number'' 
$a^\dag a \ket{n} = n\ket{n}$, the Hamiltonian is reduced to $H = \hbar
[\omega_a + (2n+1)\chi]\sigma_z/2$. This can can be viewed as a photon-number 
dependent frequency shift  in the qubit's transition frequency:
$\Delta \omega =(2n +1)\chi$. Thus, photon number discreteness can be shown from 
the discreteness of frequency shift.

\subsection{Pumping and readout}
A more precise description of the system should also include the decay terms 
(described by the coupling of the qubit and the cavity mode with reservoir 
modes), which can be calculated from master equations. To make up for the dissipation, 
a coherent external pumping field is required, described by:
\begin{equation}\label{eq:pumping}
	H_{\rf} = \hbar \varepsilon_{\rf}(a^\dag\e^{-\mi\omega_{\rf} t} + 
	a\e^{\mi\omega_{\rf} t})
\end{equation}
This is part of the bare Hamiltonian (should be added to Eq.~(\ref{eq:bare_jc})). 
In the basis of Eq.~(\ref{eq:diag_jc}) this becomes $U H_\rf U^\dag$, and is 
actually coupled to both the ``cavity'' and the ``qubit''. However, the 
frequency $\omega_\rf$ is only near resonance with $\omega_r$, making it 
effectively coupled to the cavity only. 

To measure the spectrum of the system, a sweeping signal (at the frequency 
$\omega_s$) is required. This measuring signal behaves just like 
Eq.~(\ref{eq:pumping}) except for the signal consists of pulses ($\varepsilon_s(t)$): 
\begin{equation}\label{eq:sweeping}
	H_{s} = \hbar \varepsilon_{s}(t)(a^\dag\e^{-\mi\omega_{s} t} + 
	a\e^{\mi\omega_{s} t})
\end{equation}
And the sweeping frequency $\omega_s$ range covers the effective ``qubit'' 
frequencies we are interested in $\omega_a + (2n+1)\chi$. In this way, the 
spectrum of the system can be readout by observing the reduction of the 
transmitted sweeping signal due to the coupling of the signal and the system
(Details are shown in Section~VI in \cite{blais2004cavity}).  

\section{Key results to reproduce}
We plan to reproduce the numerical result in Fig.~3 (red lines), 
and try to see if we can reproduce the difference shown in Fig.~4 for thermal 
and coherent distributions. 

\section{Methods from the class}
The method we shall use for the project includes: 
\begin{enumerate}
	\item The Jaynes-Cummings model to describe the system
	\item Quantized electromagnetic field to describe the cavity mode
	\item Semi-classical coupling of electromagnetic field (pumping signal and 
	sweeping signal) and the system
	\item Master equation to describe the dissipation and the linewidth 
\end{enumerate}

\section{Additional concepts}
To fully understand the experiment system, it is necessary to lean about the 
Josephson junction and superconducting qubit. 
\begin{equation}
	H = 4E_C (n-n_g)^2 - E_J\cos\phi;\qquad \text{with} [n,\phi] = \mi
\end{equation}
The difference of this superconducting qubit Hamiltonian and two-level system 
is discussed in the comments of Fig.~2 in \cite{schuster2007resolving}. 

To understand the transmitted amplitude in the measurement, we probably need 
to explore the couple-mode theory. 

\bibliography{photonnumber}
\end{document}
