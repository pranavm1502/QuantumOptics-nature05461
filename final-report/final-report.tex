\documentclass[11pt, notitlepage,nofootinbib]{revtex4-1}
%\usepackage[top=0.75in,bottom=0.75in,left=1in,right=1in]{geometry} 
\usepackage{amsmath,amssymb,esint} 
\allowdisplaybreaks[0]
% \usepackage[CJKbookmarks]{hyperref}
\usepackage{bm} 
\usepackage{siunitx} 
\usepackage{graphicx} 
\usepackage{color}
\usepackage{braket}
\bibliographystyle{unsrt}
\graphicspath{{figure/}}

\renewcommand*{\vec}[1]{\bm{#1}} 
\newcommand{\dif}{\,\mathrm d}
\newcommand{\rf}{\text{rf}}
\newcommand{\thm}{\text{th}}
\newcommand\mi{\mathrm{i}}
\newcommand\e{\mathrm{e}} 
\DeclareMathOperator{\Hc}{H.c.}
\DeclareMathOperator{\Tr}{Tr}

\begin{document}
\title[Report for ELE456]{Report for ELE456: \\
``Resolving photon number states in a superconducting circuit''}
\author{Ming Lyu, Elena de la Hoz Lopez-Collado}
\affiliation{Princeton Universiy}
\maketitle
This is the final-report for the course ELE456 about the project on understanding 
the paper\cite{schuster2007resolving}. 


\bibliography{photonnumber}
\end{document}
